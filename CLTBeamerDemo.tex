\documentclass[10pt,xcolor={table,dvipsnames},t,unknownkeysallowed]{beamer}
\usetheme{CLT}

\title{Presentation Title}
\subtitle{Subtitle}
\author{Author}
\institute{Department | Institution}
\date{Presentation Date}

\begin{document}

\begin{frame}
  \titlepage
\end{frame}

\section{Introduction}

\begin{frame}{Introduction}
This template is designed with the official \href{https://brand.charlotte.edu/visual-identity/logo-system}{logo} and \href{https://brand.charlotte.edu/visual-identity/color-palette}{color palette} of the University of North Carolina at Charlotte.
\newline
\newline
It is a modified version of the UC Berkeley Beamer Theme, which can be found \href{https://www.overleaf.com/latex/templates/uc-berkeley-beamer-theme/bywswngntrws}{here}. 
\end{frame}

\section {Basic Features}

\begin{frame}{Some Features}
\begin{columns}[T]
    \begin{column}{0.48\textwidth}
        Multiple columns can be utilized as needed.
    \end{column}

    \begin{column}{0.48\textwidth}
Important words can be \alert{highlighted}, \textbf{bolded}, \emph{italicized}, or \alert{\emph{\textbf{any combination of the three}}} within the text. 
    \end{column}
\end{columns}
\end{frame}

\begin{frame}{Some Features, Cont.}
Blocks are useful for bringing attention to important definitions, theorems, or other main points:
\newline
    \begin{block}{Making Lists}
        \begin{enumerate}
            \item Use \texttt{enumerate} for numbered lists.
        \end{enumerate}
        \begin{itemize}
            \item Use \texttt{itemize} for bullet points
            \item[!] See other options for lists \href{https://www.overleaf.com/learn/latex/Lists}{here}.
        \end{itemize}
    \end{block}
\end{frame}

\begin{frame}{Some Features, Cont.}
  Use \texttt{alert<+- | alert@+>} to add to a single slide, highlighting each addition as you go:
  \begin{itemize}[<+- | alert@+>]
    \item[]
    \item This is the first important point.
    \item This is the second important point.
    \item This is the final important point.
   \end{itemize}
    
\end{frame}

\section{Mathematical Notation}

\begin{frame}{Mathematical Notation}
\begin{itemize}
    \item Equations can appear in line with the text, \(x^2 + y^2 = z^2\).
    \item Or with display math mode, equations can also be set apart: \[f(x) = \frac{1}{\sigma\sqrt{2\pi}} 
  \exp\left( -\frac{1}{2}\left(\frac{x-\mu}{\sigma}\right)^{\!2}\,\right)\]
\end{itemize}

\end{frame}

\section{Tables and Figures}

\begin{frame}{A Sample Table}

This is how a basic table will appear in this template using \texttt{tabular}:
\newline
\begin{table}[h!]
\begin{center}
\begin{tabular}{c|ccc}
\hline
\tableheadrow
& Item1 & Item2 & Item3 \\
\hline
Row1 & 1 & 2 & 3 \\
Row2 & 4 & 5 & 6 \\
Row3 & 7 & 8 & 9 \\
\hline
\end{tabular}
\medskip
\caption{\label{tab:sample}A sample table.}
\end{center}
\end{table}

\end{frame}

\begin{frame}{A Sample Figure}
To insert figures, upload the image to the file tree and use \texttt{includegraphics}:
\newline
\begin{figure}
\includegraphics[width=0.45\textwidth]{tuition}
\caption{\label{fig:figure}A sample figure (original source \href{https://management.charlotte.edu/admissions/financial-information}{here}).}
\end{figure}
\end{frame}

\end{document}
